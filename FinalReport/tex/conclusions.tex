\chapter{Conclusions and Future Work} 
\section{Conclusions}
The aim of this report was to gather a substantial dataset from drone simulations, pre-process this dataset and train an ANFIS model, implement this into the drone simulator and benchmark the performance against a PID controller. Through the analysis of RMSE of test data, a tuned Sugeno Type-1 ANFIS was judged to be the best configuration for the drone control ANFIS and therefore was implemented into the simulator. Due to the inability to extrapolate, a hybrid PID-ANFIS controller was implemented into the simulator and compared against a PID controller. Overall, the Adaptive Neuro-Fuzzy Inference System (ANFIS) implementation in drone control performed better than the PID controller in scenarios where the drone has minimal change in height. 

In terms of control response, the ANFIS controller performed better than the PID controller in three out of five of the scenarios tested with a reduction in time to destination of 22.8\% in Scenario 1, 29.2\% in Scenario 2 and 23.8\% in Scenario 5. Additionally, for Scenarios 2 and 5, the computational power needed for the controller was reduced by 24.9\% in Scenario 2 and 18.9\% in Scenario 5. This decline in computational time to destination is due to the effective training and tuning of the ANFIS, meaning that the controller does not have to use calculations to determine its control response and can instead effectively and quickly use its trained fuzzy inference system to determine the control response to be sent to the quadrotor. However, the ANFIS controller performed worse than the PID controller in terms of both control response and computational time in Scenarios 3 and 4. Through the nature of these scenarios, both requiring significant relative changes in height as well as the interpretation from the control response for the error in position $z$, it is possible to conclude that the ANFIS' limitation is in scenarios where the height changes significantly. The most plausible explanation for this is the relatively high NRMSE for the pitch output at 24.1\% meaning the ANFIS produced sub-optimal responses resulting in the drone not effectively reducing its error in $z$ position. 

The computational memory results showed that the ANFIS controller required a slightly higher allocated memory usage, at an average of 5.98\% higher across all five scenarios, as well as a higher peak memory usage, at an average of 1.7\% higher across all five scenarios. This can be explained by the duality of the ANFIS controller resulting in the controller having to determine whether to use a PID or ANFIS control response and therefore requiring more computational memory. This weakness stems from the ANFIS' inability to extrapolate results due its defined membership functions. The ways to solve this could be to use a wider dataset or to use a more computationally efficient way to construct the dual ANFIS controller. 
\section{Future Work} 
Future work could look into building a more realistic and practical simulator from which data can be collected. Due to time-frame constraints, more advanced simulation softwares and simulators were not able to be built. Therefore, choosing to build the simulator in MATLAB may have been a limitation to the practicality and reliability of the dataset. In addition, future work could use a more extensive dataset to train the ANFIS controller. This would require powerful computers (e.g. a supercomputer cluster) in order to train the ANFIS with larger datasets of above \SI{1e6}{} datapoints. A more extensive dataset and reliable simulator would likely reduce the Normalised Root Mean Square Error found for the output variables and therefore improve the accuracy of the ANFIS response. Additionally, it would increase the range of parameters that the ANFIS controller would be trained on therefore be able to use the ANFIS response for more scenarios. 

Future work could also involve looking to optimise the construction of a dual ANFIS controller. As alluded to in the conclusion, this was the key factor that restricted the performance of the ANFIS Controller in this application. Hence, working to establish a more computational efficient way to utilise both control methods would potentially allow the ANFIS controller to outperform current methods. By reducing the computation required within the dual controller, the performance can be further improved. The Model Predictive Control (MPC) framework discussed in the literature review could be applied to the ANFIS in order to potentially reduce the computation. 

Finally, as this study used the MATLAB Fuzzy Logic Designer app released in the 2023a version, there are likely to be improvements to this application within MATLAB, or within other software. Utilising an advancement, such as being able to tune an ANFIS which has more than one output, would mean that the ANFIS controller would not need to be made up of four separate inference systems but rather could be integrated within a single fuzzy-inference system. This, in turn, could reduce the computational power required and provide a faster response. 
