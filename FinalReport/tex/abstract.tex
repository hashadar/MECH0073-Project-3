\section*{Abstract}
Unmanned Aerial Systems, such as quadrotors, are being increasingly used in various applications such as military, imagery and payload delivery. However, autonomous drone control is currently limited by the compromise of cost and accuracy. Current state of the art methods involve using PID controllers which use assumptions such as linearisation to calculate the appropriate control signal. Within applications like medical imaging, autonomous vehicles and drone trajectory tracking, an Adaptive Neuro-Fuzzy Inference System controller has been found to provide promising results. This report applies this approach to drone control and uses Matlab's Fuzzy Logic Designer application to make design decisions and tune the parameters of the ANFIS. The optimal configuration of the ANFIS was the Sugeno Type-1 ANFIS which provided a Normalised Root Mean Square Error of 13.2\% for the Roll control signal output, 24.1\% for the Pitch, 4.8\% for the Yaw and 19.1\% for the Thrust. The ANFIS was implemented into a dual ANFIS controller which provided a ANFIS response if the inputs were in the range and a PID otherwise. Results showed an average of 25.3\% decrease in time taken to reach its destination for three out of five test scenarios which required a relatively minimal change in height of the drone. In the other two scenarios where there was a significant change in height, the ANFIS controller was 16.5\% slower to reach its destination on average. The computational memory required by the ANFIS controller (allocated computational memory) was 5.98\% higher than the PID controller, likely due to the duality of the ANFIS controller. Therefore, the report found noticeable improvements when comparing ANFIS to PID drone control, with future work involving increasing the training dataset and computational efficiency of the duality of the controller.

Keywords: \textit{Adaptive Neuro-Fuzzy Inference System (ANFIS), Drone Control, Sugeno and Mamdani ANFIS, Proportional-Integral-Derivative Control (PID), Fuzzy Logic}
\section*{Acknowledgements}
We would like to take this opportunity to acknowledge the guidance and support of our project supervisor Dr Yuanchang Liu. We would also like to thank MAL (Massive Analytic Ltd.) for their insight and specifically Ivan Novikov for his continuous advice throughout the project. 
\section*{Git Repository}
All code used and datasets used can be found in a public GitHub repository linked \href{https://github.com/hashadar/MECH0073-Project-3}{here}. Please \href{hasha.dar.19@ucl.ac.uk}{email} Hasha Dar, if the repository is unavailable to view or for any questions. 